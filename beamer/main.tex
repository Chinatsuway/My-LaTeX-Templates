\documentclass[aspectratio=169]{beamer}

\usepackage{xeCJK}
\usetheme{Berlin}      % 使用 Berlin 主题
\usecolortheme{seagull} % 使用 seagull 配色

% 引入全局配置
\PassOptionsToPackage{a4paper, top=35mm, bottom=25mm, left=28mm, right=26mm}{geometry}
\usepackage{geometry} % 页边距设置

% 分栏设置
\setlength{\columnsep}{2em} % 分栏间距
\columnseprule=1pt % 分栏分隔线宽度

% 表格设置
\usepackage{array} % 表格扩展
\usepackage{booktabs} % 专业表格样式
\usepackage{float} % 浮动体控制
\usepackage{multirow} % 多行合并
\usepackage{graphicx} % 图片支持

% 列表设置
\usepackage{enumerate} % 枚举列表

% % 页眉页脚设置
% \usepackage{fancyhdr} % 页眉页脚控制
% \fancypagestyle{plain}{
%   \fancyhf{} % 清空页眉页脚
%   \renewcommand{\headrulewidth}{1pt} % 页眉分隔线宽度
%   \fancyfoot[C]{·~\thepage~·} % 页脚居中显示页码
%   \renewcommand{\footrulewidth}{1pt} % 页脚分隔线宽度
% }
% \pagestyle{plain} % 使用plain样式

% 参考文献设置
\usepackage{gbt7714} % 国标参考文献格式
\bibliographystyle{gbt7714-numerical} % 设置参考文献样式

% 超链接设置
\PassOptionsToPackage{colorlinks=true}{hyperref} % 超链接颜色
\usepackage{hyperref} % 超链接支持

% 行号设置
\usepackage[mathlines, switch]{lineno} % 行号支持

% 索引设置
\usepackage{imakeidx} % 索引生成
\makeindex % 初始化索引

% 随机文本
\usepackage{lipsum} % 西文随机文本
\usepackage{zhlipsum} % 中文随机文本
% \setCJKmainfont[BoldFont={SimHei},ItalicFont={KaiTi}]{SimSun} % 设置中文字体

% % 设置主字体为思源宋体
% \setCJKmainfont{Source Han Serif SC}[
%   BoldFont=Source Han Serif SC Bold,
%   ItalicFont=Source Han Serif SC Light
% ]

% % 设置无衬线字体为思源黑体
% \setCJKsansfont{Source Han Sans SC}[
%   BoldFont=Source Han Sans SC Bold
% ]

% % 设置等宽字体为系统默认
% \setCJKmonofont{SimSun}

\punctstyle{kaiming} % 设置中文标点样式

% 特殊字体效果
\usepackage{xeCJKfntef}
\xeCJKsetup{underdot/symbol={·}} % 设置下加点符号
\newcommand{\dotemph}[1]{\CJKunderdot{#1}} % 定义加点强调命令

\usepackage{xcolor}

% 定义工大蓝(C100 M70 Y0 K0)
\definecolor{NWPU-blue}{RGB}{0, 82, 155} % 工大蓝的 RGB 值

% 定义普鲁士蓝(#003153)
\definecolor{Prussianblue}{RGB}{0, 49, 83} % 普鲁士蓝的 RGB 值

% 定义某种颜色
%\definecolor{}{RGB}{0, 49, 83} % 该颜色的 RGB 值

\begin{document}

\title{My \LaTeX Beamer Template}
\subtitle{Subtitle}
\author{Name}
\date{\today}
\maketitle

\section{Introduction\\引言}
\begin{frame}
  \frametitle{Introduction\\引言}
  \lipsum[1]
  \framebreak
  \zhlipsum[1]
\end{frame}

\section{Test\\测试}
\subsection{Text\\文本}
\subsubsection{Lorem ipsum\\乱数假文}
\begin{frame}
  \frametitle{Lorem ipsum\\乱数假文}
  \lipsum[2]
  \framebreak
  \zhlipsum[2]
\end{frame}

\subsubsection{公式}
\begin{frame}
以下是一个行内公式:\( E = mc^2 \)。

以下是一个行间公式:
\[
\int_{a}^{b} f(x) \, dx = F(b) - F(a)
\]
\end{frame}

\subsection{图表}
\subsubsection{图片}
\begin{frame}
以下是一张图片:
\begin{figure}[htbp]
    \centering
    \includegraphics[width=0.5\textwidth]{example-image} % 使用 example-image 占位图
    \caption{这是一张图片}
    \label{fig:example}
\end{figure}
\end{frame}

\subsubsection{表格}
\begin{frame}
以下是一个表格:
\begin{table}[htbp]
    \centering
    \begin{tabular}{|c|c|c|}
        \hline
        序号 & 名称 & 描述 \\
        \hline
        1 & 项目A & 描述A \\
        2 & 项目B & 描述B \\
        3 & 项目C & 描述C \\
        \hline
    \end{tabular}
    \caption{这是一个表格}
    \label{tab:example}
\end{table}
\end{frame}

\subsection{超链接}
\begin{frame}
以下是一些超链接:
\begin{itemize}
    \item 访问 \href{https://www.latex-project.org/}{LaTeX 官网}。
    \item 引用图片 \ref{fig:example} 和表格 \ref{tab:example}。
    \item 引用公式 \( E = mc^2 \)。
\end{itemize}
\end{frame}

\subsection{参考文献测试}
\begin{frame}
以下是一个参考文献引用:\cite{latexcompanion}。
\framebreak
以下是参考文献列表:
\begin{thebibliography}{9}
    \bibitem{latexcompanion}
    Michel Goossens, Frank Mittelbach, and Alexander Samarin.
    \textit{The LaTeX Companion}.
    Addison-Wesley, 1993.
\end{thebibliography}
\end{frame}

\section{Conclusion\\结论}
\begin{frame}
  \frametitle{Conclusion\\结论}
  \lipsum[3]
  \framebreak
  \zhlipsum[3]
\end{frame}

\end{document}